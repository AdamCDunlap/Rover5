\documentclass{article}

\usepackage{amsfonts}


\begin{document}

\title{Determining Current Position of a Mecanum Robot From Wheel Speeds Over Time}
\author{Adam Dunlap}
\date{\today}
\maketitle

\section{Definitions}
\begin{itemize}
\item Let $v$ be a function that takes $\mathbb{R}$ --- a time --- and returns $\mathbb{R}^4$ --- the velocity of each wheel obtained by sensor data.

\item Let $r$ be a function that takes $\mathbb{R}^4 * \mathbb{R}$ --- the velocity of each wheel and the abosolute angle of the robot --- and returns $\mathbb{R}^2 * \mathbb{R}$ --- the velocity of the robot and the rotational velocity of the robot.

\item Let $n$ be a function that takes $\mathbb{R}^4$ --- the velocity of each wheel --- and returns $\mathbb{R}$ --- the rotational velocity of the robot.

\item Let $k$ be a function that takes $\mathbb{R}$ --- a time --- and returns $\mathbb{R}^2 * \mathbb{R}$ --- the velocity of the robot and the rotational velocity of the robot

\end{itemize}

\section{Forward Kinematics}
The goal of forward kinematics is to find $k$ --- where the robot is given a time. This can be found with
\[
k(t) = \left(r\left(v\left(t\right), \int_0^t n\left(x\right) dx\right), n\left(t\right)\right)
\]

\end{document}
